%Vorlage
\documentclass[12pt,a4paper]{scrartcl}
\usepackage[english]{babel} %Für die indirekte Angabe von Umlauten. Es müssen dann Umlaute wie folgt im Code angegeben werden: "a "o "u "s.

\usepackage[utf8]{inputenc}
%Math
\usepackage{amsmath, amsthm, amssymb}
\usepackage{braket}
\newcommand{\tens}[1]{% https://tex.stackexchange.com/questions/171788/always-have-the-ring-of-the-tensor-product-below-the-otimes -> Tensor Product
  \mathbin{\mathop{\otimes}\displaylimits_{#1}}%
}
%Page numbers
\usepackage{enumerate}
%Graphics
\usepackage{graphicx}
\usepackage{array}% http://ctan.org/pkg/array
\usepackage{floatrow}
\graphicspath{{./images/}}
%Quantum circuits http://mirrors.ibiblio.org/CTAN/graphics/pgf/contrib/quantikz/quantikz.pdf
\usepackage{tikz}
\usetikzlibrary{quantikz}
\usepackage{lscape}
\usepackage{setspace}
\onehalfspacing
\usepackage{wrapfig}
\usepackage{hyperref}% für die Einbettung von Hyperlinks
\def\UrlBreaks{\do\/\do-}
\usepackage{multirow}
\usepackage{csquotes} %Quotations
%Code
\usepackage{pythonhighlight}


% Margins
%\usepackage{geometry} % Document Margins
%\setlength{\topmargin}{0cm}
%\setlength{\parindent}{5mm}
%\setlength{\parskip}{2mm}
%\setlength{\evensidemargin}{0mm}
%\setlength{\oddsidemargin}{0cm}
\pagestyle{headings}



\begin{document}
\thispagestyle{empty}
\vspace*{-3cm}
\begin{center}
\large \textsc{Bern University of Applied Sciences}
\vspace{0.5cm}
\hrule
\vspace{4.5cm}
{\Large \textsc{Written Report\\
Bachelor Thesis}}\\
{\large HS 2020/21}\\
\vspace{1cm}
{\Large \bfseries
Generation of synthetic ground truth for ophthalmic medical image analysis}\\

\vspace*{1cm}
{\large Tutor:  Prof. Dr. Tiziano Ronchetti}
\end{center}
\vspace*{1cm}

\begin{abstract}
Machine Learning framework that process ophthalmic OCT images and predicts synthetic vitreal, retinal, choroidal and scleral semantic segmentations based on a combination of Image Augmentation techniques and a Convolutional Neural Network presenting an encoder-decoder structure. After cross-validation the results indicate that our framework can automatically segment a regular OCT scan with a mean Jaccard coeficient of 96,97\%, showcasing a hamming distance of xx\% when compared to human ophthalmologists. Furthermore, the authors describe a detailed workflow involving the utilisation of state of the art machine learning techniques with comprehensive explanations on the data preparation, model implementation and evaluation criteria.  
\end{abstract}

\vspace{2cm}
\hspace*{5.2cm}
\parbox{8.2cm}

\begin{tabular}{ll}

Submitted by: & Emeline Liebeherr\\
& Rayner Zorrilla Alfonzo\\

Submission deadline: & Thursday, June 17th, 2021


\end{tabular}

\newpage
\pagenumbering{Roman}
\tableofcontents

\newpage
\pagenumbering{arabic}
%Und nun kommen wir zur Arbeit und fangen an die Seiten mit Arabischen Zahlen zu zählen

\section{TODO Introduction}\label{s:introduction} 
Computer vision methods for the automated segmentation of inner retinal and choroidal layers obtained from optical coherence tomography (OCT) imaging \cite{Ronchetti2019statistic} can be used to generate a reliable set of annotated scans. Such annotations can be used to develop a machine learning model to accurately detect retinal (and choroidal) thickness changes.  This provides researchers and clinicians with reliable and factual information that can help in the diagnosis and monitoring of various ocular disorders and diseases. \\

The contribution of this project lies in the documentation and explanation of the most frequent methods used for measuring retinal thickness from an algorithmic perspective in order to provide a framework to easily extract features from OCT B-scans. The report is structured as follows. Section 2 presents an introduction to the medical background needed in order to understand the retinal structure, functions and the importance of retinal layer thickness measurement. Section 3 will offer an overview of the most common techniques used in the industry for carrying out retinal layer segmentation from OCT scans. Section 4 will offer the implementation and results of the methods discussed. Finally, section 5 and 6 will respectively offer a conclusion and an overview of the future work to be carried out by the authors. \\

\subsection{Motivation}
The choroid is one of the layers of the eyeball wall, located between the sclera on the outside and the retina on the inside. The choroid is responsible for the irrigation and circulation of the ocular metabolism in order to supply the outer retina with oxygen and metabolites and is therefore highly vascularised. X2

Its thickness can depend on several factors, including age. It has been found that the thickness of the choroid decreases with age, however several recent research studies on choroidal development during childhood and adolescence contradict this finding. In particular, subfoveal choroidal thickness has been found to be negatively correlated in Asian children, where the prevalence of myopia is higher. Longitudinal studies of adolescents have shown that the eyeball lengthens during the development of myopia. In the case of severe myopia this process is associated with a significant thinning of the choroidal thickness. Therefore the following markers are strongly considered for monitoring the progression of myopia and for predicting myopia: choroidal thickness and structure.

The main challenge in detecting disease progression is to detect even small changes as early as possible. Optical coherence tomography (OCT) imaging allows the capture of highly resolved details of the retina and choroid to detect minute changes in the structure of the retina. X1

In order to be able to detect changes in choroidal thickness it is necessary to be able to detect the different layers of the eyeball wall and their borders from OCT scans in order to compare results over several years. 
However, a manual approach raises two important problems: firstly, the number of scans to be processed (layer identification) is considerable, secondly, the high vascularity of the choroid makes it difficult to distinguish its border with the sclera.

The aim of this project is to develop a Deep Learning model capable of processing  ophthalmic  OCT  images  and predicts  synthetic  vitreal,  retinal,  choroidal  and  scleral  semantic  segmentations  based  on  a  combination  of  Image  Augmentation  techniques  and  a Convolutional Neural Network presenting an encoder-decoder structure with an accuracy of at least XX\%. This will automate the measurements made on the scans and considerably reduce the time needed to produce a result, thus reducing the workload of the experts and ensuring a certain accuracy in the detection of the choroid. 


\subsection{Goal}
\subsection{Note on the document and work}
\subsubsection{Tools version}
The program at its current state is dependent on the following frameworks : 
    
    \begin{table}[]
    \begin{tabular}{ll}
    Framework & Version \\
    Keras     & 2.4.3   \\
    Opencv-python  &   4.5.1.48 \\
    Python & 3.8.5 \\
              &        
    \end{tabular}
    \end{table}

\subsubsection{Document organisation}

\section{Medical Background}\label{s:medical_background}

The retina is a layer of tissue situated in the periphery of the ocular globe. As a part of the central nervous system, the retina "converts the graded electrical activity of photoreceptors into action potentials that travel to the brain via axons in the optic nerve" \cite{purves2001}. In other terms, it converts light into electric signals that are sent directly to the brain and are interpreted as images. \\

Measuring thickness changes in the retina allows to detect the evolution of certain diseases like Diabetes \cite{Jiang2018}, Alzheimer, Glaucoma and other neurodegenerative diseases \cite{DENHAAN2017162}. For example, in the case of Alzheimer is has been detected in \cite{DENHAAN2017162} that both measuring methods, Mean Peripapillary Retinal Nerve Fiber Layer (RNFL) and macular thickness show a consistent retinal thinning for diagnosed patients when compared to healthy individuals \cite{DENHAAN2017162}.\\

The process of measuring retinal thickness involves taking a cross sectional scan of the retina and anterior segment, with the help of an imaging procedure known as Optical Coherence Tomography (OCT). Ocular Bidimensional OCT, is a standard, non-invasive procedure in the field of ophthalmology and consist on a micrometer resolution scan of a 2D eye section obtained by emitting low coherence light waves which allow high resolution longitudinal imaging of the corneal, retinal and chroroidal structures \cite{Ronchetti2019statistic}. Locating the retina within a Bidimensional OCT scan (B-Scan) involves the detection of the retina's two boundaries, the Inner Limiting Membrane (ILM) and the Bruchs Membrane (BM). On one side, the ILM serves as a barrier between the retina and the vitreous body \cite{MACNAIR2015343} while the BM divides the retina from the choroid \cite{BOOIJ20101}. The retinal thickness is defined as the average of all Euclidean/vertical distance in the $z$ axis direction between the ILM and BM boundaries \cite{Ronchetti2019statistic}. Another approach to measuring thickness is to define 5 symmetrical regions with respect to the vertical direction, we will name this regions A (subfoveal), B (parafoveal) and C (perifoveal), the thickness values for such regions are provided in the form of the mean of all height differences within a particular section A, B or C respectively \cite{Ronchetti2019statistic} (See Figure \ref{fig:final_rendering}). 

\section{Technical Background}\label{TechBack}

\subsection{OCT}
There are several approaches for detecting retinal boundaries in OCT B-scans, among them we can cite pixel intensity variations, texture analysis and graph search based segmentation techniques. Pixel intensity variations are often used when calculating the total retinal thickness \cite{Alonso-Caneiro2013}, the method involves a series of computer graphic processing steps where each pixel of the image is compared to a threshold chosen according to the color intensity of the pixels that contain the information about the location of the boundary \cite{Fabritius:09}. In the case of the ILM, it can be easily extracted because the lens pixels in the OCT scans are mostly black while the membrane's pixels are mostly white, this is mainly due to the density of the element, the more dense the element is, the whiter it will appear in the image \cite{Brar597}. Therefore, taking the image from the top left corner, one can run a search algorithm that finds the first white pixel \footnote{The first pixel that is over the threshold} from top to bottom and return its position. Running this same algorithm across the $x$ axis produces us an array of pixels that can be smoothed by using cubic-spline interpolations that corresponds to the ILM. The process for extracting the BM inner-most layer is the same, since the sclera presents opaque pixels that can be omitted by the algorithm, the process would start from the bottom left corner and runs the algorithm from bottom to top searching white pixels until it reaches the BM.  
\subsection{Convolutional Neural Networks}
\subsection{U-NET}


\section{Previous work}\label{prevWork}
\subsection{Computer vision methods for detection of early choroidal thickness changes in myopic Asian school children}

\section{Discussion}\label{Discussion}

\subsection{Data Collection}
\subsection{Data Preparation}
\subsubsection{Mask segmentation}
\subsection{Model Development}
\subsection{Model Training}
\subsection{Model Evaluation}

\section{Results}\label{Results}


\newpage
\section{Conclusion}


\section{Future Work}

\markboth{}{}

\newpage

\bibliographystyle{plain} % We choose the "plain" reference style
\bibliography{bibli} % Entries are in the "bibi.bib" file




\newpage
\thispagestyle{empty}
\markboth{}{}
  \normalsize
\begin{center}
\huge{\textbf{ Declaration of Independence}}\\[40mm]
\end{center}
\large
We confirm that the above work has been produced by the authors without any unauthorized assistance and without the use of any other means than those indicated, and that I have marked as such all passages that have been taken literally or meaningfully from published or unpublished writings.\\[50mm]
Bienne, the \today

\newpage



\end{document}